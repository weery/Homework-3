\subsection*{a)}

We want to study the time and space evolution of the two interacting populations $S$ (susceptibles) and $I$ (infected) that adheres the following dynamics,

\begin{equation}
\label{eq:partSt}
\frac{\partial S}{\partial t} = b(S+I)-cS -\frac{S(I+S)}{K}-aSI + D\nabla^2S,
\end{equation}

\begin{equation}
\label{eq:partIt}
\frac{\partial I}{\partial t}= -cI -\frac{I(I+S)}{K}+aSI +D\nabla^2I.
\end{equation}

For the spatially homogeneous model, that is $D=0$, we find the steady states by solving,

$$
\frac{\partial S^*}{\partial t}=b(S^*+I^*)-cS^*-\frac{S^*(I^*+S^*)}{K}-aS^*I^* =0
$$

and

$$
\frac{\partial I^*}{\partial t}=-cI^*-\frac{I^*(I^*+S^*)}{K}+aS^*I^*=0.
$$

We can directly rule our the steady states which involve $S^*\leq0$ or $I^*<0$ since they bear no biological interest.

\begin{itemize}
\item $S^*_1=K(b-c), \;\; I^*_1=0$
\item $S^*_2=\frac{b}{a}, \;\; I^*_2=K(b-c)-\frac{b}{a}$ 
\end{itemize}

To see the stability of these steady states we need to consider the Jacobian for theses states,

\begin{equation}
\mathbb{J}^*_n=\left.\left(
\begin{array}{cc}
\frac{\partial f}{\partial S} & \frac{\partial f}{\partial I} \\
\frac{\partial g}{\partial S} & \frac{\partial g}{\partial I}
\end{array}\right)\right|_{(S^*_n,I^*_n)},
\end{equation}

where 

\begin{equation}
f(S,I)=b(S+I)-cS-\frac{S(I+S)}{K}-aSI
\end{equation}

\begin{equation}
g(S,I)=-cI-\frac{I(I+S)}{K}+aSI.
\end{equation}

So
\begin{equation}
\mathbb{J}^*_n=\left.\left(
\begin{array}{cc}
b-c-\frac{I+2S}{K}-aI & b-\frac{S}{K}-aS \\
-\frac{I}{K} +aI & -c -\frac{2I+S}{K}+aS
\end{array}\right)\right|_{(S^*_n,I^*_n)},
\end{equation}

To have a stable steady state we need that 

\begin{enumerate}
\item $Tr(\mathbb{J}^*_n)<0$
\item $Det(\mathbb{J}^*_n)>0$
\end{enumerate}


For the first steady state we get that the steady state is stable for

$$
K<\frac{2b-c}{a(b-c)}=\frac{b}{a(b-c)}+\frac{1}{a(b-c)}
$$$$
K<\frac{b}{a(b-c)}.
$$
For the second steady state, the condition for being stable is fulfilled, also both I and S have positive nonzero values for a critical value $K_c$ as follows
$$
K_c>\frac{c}{a(b-c)}
$$$$
K_c>\frac{b}{a(b-c)}
$$

and since $b>c$ we have that the critical value of $K_c$ is
\begin{equation}
k_c=\frac{b}{a(b-c)}.
\end{equation}
\subsection*{b)}

The introduction of the new variable $N=I+S$, which is the total population, inserted into equations \eqref{eq:partIt} and \eqref{eq:partSt} gives,

\begin{equation}
\frac{\partial N}{\partial t}=bN -cN -\frac{N^2}{K}+D\nabla^2N.
\end{equation}

If we assume that we are studying the dynamics in one dimension, we can replace the $\nabla^2$ operator with a second derivative in that direction, i.e. $\nabla^2\rightarrow \frac{\partial^2}{\partial x^2}$ and we get the time evolution as,

\begin{equation}
\frac{\partial N}{\partial t}=(b-c)N\left(1-\frac{N}{K}\right)+D\frac{\partial^2 N}{\partial x^2}.
\end{equation}

To see if the dynamics exhibits a travelling wave solution we try to reduce the spatial and temporal dimension to one by $z=x-ct$, and changing the population variable to $N(x,t)\rightarrow u(z)$ which would yield,

\begin{equation}
-c\frac{\partial u}{\partial z}=(b-c)u\left(1-\frac{u}{K}\right)+D\frac{\partial^2 u}{\partial x^2}.
\end{equation}

To study the dynamics of this system introduce yet another variable, $v=\frac{\partial u}{\partial z}$, and receive the system 

\begin{align*}
\frac{\partial u}{\partial z}&=v\\
\frac{\partial v}{\partial z}&=-\frac{c}{D}v-\frac{(b-c)u(1-\frac{u}{K})}{D}.
\end{align*}

This system has the steady states $(u^*,v^*)=(0,0)$ and $(u^*,v^*)=(K,0)$.

To check the stability of these steady states we need to consult the Jacobian matrix,

\begin{equation}
\mathbb{J}=\left(\begin{array}{cc}
0 & 1 \\
  \frac{(b-c)(K-2u)}{KD} & -\frac{c}{D}
\end{array}\right),
\end{equation}

 evaluated at the steady states. To check the stability of the steady states we need to check the signs of the determinant and trace of the jacobian, and the value of the determinant for these states are,
 
 $$
 det(\mathbb{J}^*_1)= -\frac{b-c}{D}<0
 $$
 and
 
 $$
 det(\mathbb{J}^*_2)=\frac{b-c}{D}>0.
 $$
 
 Their traces does not depend on the steady state itself and they are therefore the same 
 $$
 Tr(\mathbb{J}^*_1)=Tr(\mathbb{J}^*_2)=-\frac{c}{D}<0.
 $$
 
The first state has $det(\mathbb{J}^*)<0$ and $Tr(\mathbb{J}^*)<0$ which makes it a candidate for being both a saddle point and a stable spiral. It is a saddle point if $c^2_s>4D(b-c)$ and a steady spiral if $c^2_s<4D(b-c)$. To get a travelling wave solution we need to go from the steady state with a low amount of population, i.e. the first steady state, to the second one. To accomplish this we need the first steady state to be a saddle point, otherwise there is no way for the population to increase.\\
So now we have an expression for the velocity $c_s$ when we have a saddle point and the minimum velocity is thusly,

$$
 c^2_s>4D(b-c)\Rightarrow c_s > \sqrt{4D(b-c)}.
 $$
 
\subsection*{c)}

\begin{figure}
\centering
\includegraphics[scale=0.5]{img/listdensityplot_S.png}
\includegraphics[scale=0.5]{img/listdensityplot_P.png}
\caption{\label{fig:pic1c} In the left panel, depicting the development of the susceptible individuals $S$ in time and space, the speed of the travelling wave can calculated by $c=\frac{\Delta x}{\Delta t}$. The same can be done for the infected individuals $I$ in the right panel.}
\end{figure}

We can see the expansion of the susceptible $S$ in the left panel of figure \ref{fig:pic1c}. The population expands in both directions with equal constant velocity and expansion resembles a travelling wave. The steady state for the spatially inhomogeneous susceptible habitat size are the same as the homogeneous steady stable state which is $10$. The speed of this travelling wave is estimated to $1.35$. We see in the figure that the number of susceptible overshoots over the steady state, which means that the spatial term trumps the maximum capacity from the carrying capacity term until the number of infected has time to kill of the superfluous population. \\

We see that the number of infected does not overshoot, but stops at the steady state $I^*=5$. We can also calculate the velocity to $1.33$ which is just below the velocity for the susceptible. This might be because the number of infected depends on the number of susceptible of the previous batch and have a harder time to catch up to the number of susceptible due to the limited locality.

